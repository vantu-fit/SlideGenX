\section{Ý Tưởng Và Mục Tiêu Đề Tài}

\subsection{Ý tưởng dự án}
Xây dựng một ứng dụng nhắn tin cho phép hai người dùng, dù sử dụng hai ngôn ngữ khác nhau, vẫn có thể giao tiếp một cách tự nhiên và hiểu nhau ngay lập tức. Cốt lõi của ứng dụng là một mô hình ngôn ngữ lớn tích hợp, có khả năng tự động phát hiện ngôn ngữ và dịch thuật tin nhắn trong thời gian thực giữa các ngôn ngữ của hai bên tham gia trò chuyện.

\subsection{Minh họa cụ thể}
\begin{itemize}
    \item Người dùng A (ngôn ngữ mẹ đẻ: Tiếng Việt) gửi tin nhắn ``Xin chào, bạn khỏe không?''
    \item Hệ thống dựa trên ngôn ngữ của người dùng A là Tiếng Việt.
    \item Người dùng B (ngôn ngữ mẹ đẻ: Tiếng Pháp) nhận được tin nhắn đã dịch ``Bonjour, comment allez-vous ?''
    \item Khi Người dùng B trả lời bằng Tiếng Pháp ``Je vais très bien, merci !'', Người dùng A sẽ nhận được bản dịch ``Tôi rất khỏe, cảm ơn bạn!'' bằng tiếng Việt.
\end{itemize}
Quá trình này diễn ra hoàn toàn minh bạch đối với người dùng, tạo cảm giác như đang trò chuyện trực tiếp bằng cùng một ngôn ngữ.

\subsection{Mục tiêu đề tài}
\begin{itemize}
    \item \textbf{Xóa bỏ rào cản ngôn ngữ:} Cung cấp một công cụ hiệu quả giúp người dùng từ các quốc gia và nền văn hóa khác nhau dễ dàng kết nối và tương tác.
    \item \textbf{Nâng cao trải nghiệm người dùng:} Mang đến trải nghiệm nhắn tin mượt mà, nhanh chóng và tự nhiên, không bị gián đoạn bởi quá trình dịch thuật thủ công.
    \item \textbf{Thúc đẩy giao tiếp đa văn hóa:} Tạo điều kiện cho sự hiểu biết và hợp tác toàn cầu thông qua giao tiếp không giới hạn.
    \item \textbf{Khám phá tiềm năng của LLMs trong ứng dụng thực tế:} Chứng minh khả năng của các mô hình ngôn ngữ lớn trong việc giải quyết các vấn đề phức tạp trong lĩnh vực dịch máy.
\end{itemize}